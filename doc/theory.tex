\documentclass[]{article}
\usepackage[left=2cm,top=2cm,right=2cm]{geometry}
\begin{document}

\title{Theory used by libgfshare}
\author{Simon McVittie}
\date{23rd April 2006}
\maketitle

\section{Introduction}

libgfshare implements Shamir secret sharing [SHAMIR] over the field $GF(2^8)$,
instead of $GF(p)$ for a prime $p$ as suggested by Shamir's paper.
This document aims to prove the security and integrity of this scheme.

Note that while I believe this document to be correct, I accept no
responsibility for loss or damage caused by relying on the correctness
of my proof.

\section{Definitions}

Let $F$ be a field with multiplicative identity 1 and additive identity 0.

If $A = \{(a_1, b_1), \cdots, (a_n, b_n)\}$, with the $a_i$ distinct nonzero
elements of F and the $b_i$ elements of $F$, indexed by $I = \{1,\cdots,n\}$,
then define

\[
        P_A(x) = \sum_{j\in I} {b_j \prod_{k\in I, k\neq j} {(x-a_k)(a_j-a_k)^{-1}}}
\]

a polynomial of degree at most $n-1$. (By distinctness of the $a_i$, the
inverses required exist.) This is the Lagrange interpolating polynomial
for the points in $A$.

\section{Lemma 1}

Let $a_1,\cdots,a_t\in F$ be distinct and nonzero; let
$b_1,\cdots,b_{t-1},c\in F$ be arbitrary. Then there exists $b_t\in F$
such that if $A = \{(a_1, b_1), \cdots, (a_t, b_t)\}$ then $P_A(0) = c$.

\subsection{Proof}

Let $I = \{1, \cdots, t\}$. We have

\[
P_A(0) = \sum_{j\in I} {b_j \prod_{k\in I, k\neq j}{-a_k(a_j-a_k)^{-1}}}
= \sum_{j\in I} {y_j \prod_{k\in I, k\neq j}{a_k(a_k-a_j)^{-1}}}
\]

Let

\[
b_t = \left[c + \sum_{j\in I, j\neq t}{b_j\prod_{k\in I,k\neq j}{a_k(a_j-a_k)^{-1}}}\right]
        \left[\prod_{k\in I,k\neq t}{a_k^{-1}(a_k-a_t)}\right]
\]

Then

\[
P_A(0) = \sum_{j\in I, j\neq t} {b_j \prod_{k\in I, k\neq j}{a_k(a_k-a_j)^{-1}}} + b_t \prod_{k\in I, k\neq t}{a_k(a_k-a_t)^{-1}}
\]\[
= \sum_{j\in I, j\neq t} {b_j \prod_{k\in I, k\neq j}{a_k(a_k-a_j)^{-1}}} - \sum_{j\in I, j\neq t} {b_j \prod_{k\in I, k\neq j}{a_k(a_k-a_j)^{-1}}} + c
\]\[
= c
\]

as required.

\section{Lemma 2}

For any $x_1,\cdots,x_t$ distinct and nonzero elements of $F$, and any
$y_1,\cdots,y_t,u$ arbitrary elements of $F$, let

\[X = \left\{(x_1,y_1),\cdots,(x_t,y_t)\right\}\]

and

\[U = \left\{(x_1,y_1),\cdots,(x_{t-1},y_{t-1}),(u,P_X(u))\right\}\]

Then $P_X = P_U, i.e. P_X(x) = P_U(x)$ for all $x\in F$.

\subsection{Proof}

Let $S_{a,b} = \left\{(x_1,y_1),\cdots,(x_{t-1},y_{t-1}),(a,b)\right\}$.
Then

\[
P_{S_{a,b}}(x) = \sum_{j<t} y_j (x-a)(x_j-a)^{-1}
        \prod_{k\ne j,k<t}(x-x_k)(x_j-x_k)^{-1}
        + b\prod_{k<t}(x-x_k)(a-x_k)^{-1}
\]

Hence if we let $d_{i,j} = x_i - x_j$ and $e_i = u - x_i$ (both of which
are necessarily nonzero, by distinctness of the $x_i$ and $u$) we have

\[
P_X(u) = \sum_{j<t} y_j e_t d_{j,t}^{-1}
        \prod_{k\ne j,k<t}e_kd_{j,k}^{-1}
        + y_t\prod_{k<t}e_kd_{t,k}^{-1}
\]

and if we also let $f_i = x - x_i$,

\[
P_U(x) = \sum_{j<t} y_j (u-x)e_j^{-1} \prod_{k\ne j,k<t}f_k d_{j,k}^{-1}
    + P_X(u)\prod_{k<t} f_k e_k^{-1}
\]\[
P_U(x) = \sum_{j<t} y_j (u-x)e_j^{-1} \prod_{k\ne j,k<t}f_k d_{j,k}^{-1}
    + \left\{\prod_{k<t} f_k e_k^{-1}\right\}\left\{
        \sum_{j<t} y_j e_t d_{j,t}^{-1}\prod_{l\ne j,l<t}e_kd_{j,l}^{-1}
        + y_t\prod_{l<t}e_ld_{t,l}^{-1}
    \right\}
\]

Expanding,

\[
\begin{array}{rcl}
P_U(x) & = & \sum_{j<t}y_j(u-x)e_j^{-1}\left\{\prod_{k\ne j,k<t}f_kd_{j,k}^{-1}\right\} \\
& & + \sum_{j<t}y_je_td_{j,t}^{-1}
        \left\{\prod_{l\ne j,l<t}e_kd_{j,l}^{-1}\right\}
        \left\{\prod_{k<t}f_ke_k^{-1}\right\} \\
& & + y_t\left\{\prod_{k<t}e_kd_{t,k}^{-1}f_ke_k^{-1}\right\}
\end{array}
\]

\[
P_U(x) = \sum_{j<t}y_j\left[
        (u-x)e_j^{-1}\left\{\prod_{k\ne j,k<t}f_kd_{j,k}^{-1}\right\}
        + e_td_{j,t}^{-1}f_je_j^{-1}\left\{
            \prod_{k\ne j,k<t}e_kd_{j,k}^{-1}f_ke_k^{-1}
        \right\}
    \right]
    + y_t\prod_{k<t}d_{t,k}^{-1}f_k
\]\[
= \sum_{j<t} \left[y_j \prod_{k\ne j,k<t}f_kd_{j,k}^{-1}\right]
        \left[(u-x)e_j^{-1} + e_te_j^{-1}d_{j,t}^{-1}f_j\right]
        + y_t\prod_{k<t}d_{t,k}^{-1}f_k
\]

Now

\[
(u-x)e_j^{-1} + e_te_j^{-1}d_{j,t}^{-1}f_j
    = (e_j^{-1}d_{j,t}^{-1})\left[(u-x)d_{j,t} + e_tf_j\right]
\]\[
    = (e_j^{-1}d_{j,t}^{-1})\left[(u-x)(x_j-x_t) + (u-x_t)(x-x_j)\right]
\]\[
    = (e_j^{-1}d_{j,t}^{-1})(x - x_t)(u - x_j)
\]\[
    = d_{j,t}^{-1}f_t
\]

Hence

\[
P_U(x)
= \sum_{j<t} \left[y_j \prod_{k\ne j,k<t}f_kd_{j,k}^{-1}\right]
        \left[f_td_{j,t}^{-1}\right]
        + y_t\prod_{k<t}d_{t,k}^{-1}f_k = P_X(x)
\]

as required.

\section{Construction}

Let $s$ be the number of ``shares'' and $t$ be the required threshold
to recover the shared secret (i.e. we construct a ``$t$ of $s$'' share).

Given a secret $f\in F$ we may construct a Lagrange interpolating
polynomial $P_X$ of degree no more than $t-1$, with $P_X(0) = f$, as
follows:

- choose distinct nonzero $x_1,\cdots,x_s \in F$

- choose arbitrary (and unpredictable) $y_1,\cdots,y_{t-1} \in F$

- use Lemma 1 to select $y_t$ such that $X = \{(x_1,y_1),\cdots,(x_t,y_t)\}$
  has the desired intercept $f$

To obtain additional shares, calculate
$y_{t+1} = P_X(x_{t+1}),\cdots,y_s=P_X(x_s)$.

\section{Alternate construction, as used in libgfshare}

In libgfshare the construction used is as follows:

- construct a polynomial $P$ by choosing arbitrary and unpredictable
  coefficients of $x,\cdots,x^{t-1}$ from $F$, and setting the coefficient
  of $x^0$ to $f$: this therefore has the desired intercept $f$

- choose distinct nonzero $x_1,\cdots,x_s \in F$ and evaluate
  $y_1 = P(x_1),\cdots,y_s = P(x_s)$

\subsection{Proof of equivalence in a finite field $F$}

Suppose $F$ is finite, as is the case in libgfshare, and that in each
construction, arbitrary choices are made from among all possible
values in $F$.

In the alternate construction, given $x_1,\cdots,x_t,f$ we choose a
polynomial $P(x) = f+m_1x+\cdots+m_{t-1}x^{t-1}$
by choosing arbitrary coefficients $m_1,\cdots,m_{t-1}\in F$, i.e.
choosing arbitrarily from among the $\left|F\right|^{t-1}$ distinct
polynomials of degree no more than $t-1$ with intercept $f$.

In the first construction, given $x_1,\cdots,x_t,f$ we obtain a polynomial
by choosing arbitrary $y_1,\cdots,y_{t-1}\in F$. The polynomials chosen
are necessarily distinct since no polynomial can pass through both $(x_i, p)$
and $(x_i, q)$ for any $p \ne q$, so by choosing each $y_i$ from among the
$\left|F\right|$ elements of $F$, we choose arbitrarily from a set of
$\left|F\right|^{t-1}$ distinct polynomials whose intercepts are all $f$.

Since there are only $\left|F\right|^{t-1}$ such polynomials, each
construction chooses arbitrarily from among the same set, and by the
pigeonhole principle there exists a bijective mapping between sets of
arbitrary $y$ values in the first construction and sets of arbitrary
coefficients in the second.

\section{Theorem: With at least $t$ pieces the secret is recoverable}

Let $B \subset\left\{(x_1,y_1),\cdots,(x_s,y_s)\right\}$ with $|B| = t$.
Then $P_B(0) = c$.

Further, if $B^\prime \subset\left\{(x_1,y_1),\cdots,(x_s,y_s)\right\}$
with $|B^\prime| > t$, then for every subset $B$ of $B^\prime$ with
$|B| = t$, $P_B(0) = f$.

\subsection{Proof}

The second part is trivially implied by the first.

Recall that $X = \left\{(x_1,y_1),\cdots,(x_t,y_t)\right\}$ and that
$P_X(0) = f$. If $B = X$ the result is true. If not, repeatedly apply
Lemma 2 to replace an element of $X$ not in $B$ with an element of $B$ not
in $X$, preserving the value of $P(0)$.

\section{Theorem: With fewer than $t$ pieces no information is gained}

Let $C \subset\left\{(x_1,y_1),\cdots,(x_s,y_s)\right\}$ with $|C|< t$.
Then for each $d\in F$, there exists $D\supset C$, $|D| = t$, such that
$d = P_D(0)$.

(In other words, any $d\in F$ remains a possible value for the secret, so
an attacker with fewer than $t$ shares has gained no information.)

\subsection{Proof}

Let $a_i$, $b_i$ be such that $C = \left\{(a_1,b_1),\cdots,(a_n,b_n)\right\}$,
some $n < t$. Choose arbitrary $a_{n+1},\cdots,a_{t}$ and arbitrary
$b_{n+1},\cdots,b_{t-1}$. Let $b_t$ be chosen by applying Lemma 1
with $c := d$. Then by choice of $b_t$, $P_C(0) = d$ as required.

\section{Implementation in $GF(2^8)$}

The program \texttt{test\_gfshare\_isfield}, compiled and run by
\texttt{make check}, demonstrates that the calculations done by
libgfshare are indeed performed in a field.

\section{Attacks not addressed}

This document has not addressed the following:

- Attacks based on the use of a predictable or partially predictable
  pseudorandom number generator might be possible.

- In the implementation used in libgfshare, the field $F$ is the field
  of byte values, with addition being bitwise exclusive-or, and multiplication
  as usual; each byte of the secret is shared separately by applying this
  algorithm separately. This means that when a secret file is shared,
  the length in bytes of each share equals the length in bytes of the
  secret. If the length of the secret is itself secret, it should be
  padded to some standard length before sharing.

\section{References}

[SHAMIR] Adi Shamir, "How to share a secret", Communications of the ACM, 22(1), pp612–613, 1979. Available at http://www.cs.tau.ac.il/~bchor/Shamir.html

\section{Copyright and disclaimer}

Copyright 2006 Simon McVittie <smcv pseudorandom co uk>.

Permission is hereby granted, free of charge, to any person obtaining a copy
of this software and associated documentation files (the "Software"), to deal
in the Software without restriction, including without limitation the rights
to use, copy, modify, merge, publish, distribute, sublicense, and/or sell
copies of the Software, and to permit persons to whom the Software is
furnished to do so, subject to the following conditions:

The above copyright notice and this permission notice shall be included in
all copies or substantial portions of the Software.

THE SOFTWARE IS PROVIDED "AS IS", WITHOUT WARRANTY OF ANY KIND, EXPRESS OR
IMPLIED, INCLUDING BUT NOT LIMITED TO THE WARRANTIES OF MERCHANTABILITY,
FITNESS FOR A PARTICULAR PURPOSE AND NONINFRINGEMENT.  IN NO EVENT SHALL THE
AUTHORS OR COPYRIGHT HOLDERS BE LIABLE FOR ANY CLAIM, DAMAGES OR OTHER
LIABILITY, WHETHER IN AN ACTION OF CONTRACT, TORT OR OTHERWISE, ARISING FROM,
OUT OF OR IN CONNECTION WITH THE SOFTWARE OR THE USE OR OTHER DEALINGS IN
THE SOFTWARE.

\end{document}
